% Created 2016-10-18 Tue 16:57
\documentclass[11pt]{article}
\usepackage[utf8]{inputenc}
\usepackage[T1]{fontenc}
\usepackage{fixltx2e}
\usepackage{graphicx}
\usepackage{longtable}
\usepackage{float}
\usepackage{wrapfig}
\usepackage{soul}
\usepackage{textcomp}
\usepackage{marvosym}
\usepackage{wasysym}
\usepackage{latexsym}
\usepackage{amssymb}
\usepackage{hyperref}
\tolerance=1000
\providecommand{\alert}[1]{\textbf{#1}}

\title{studyguide}
\author{Shantanu Vyas}
\date{\today}
\hypersetup{
  pdfkeywords={},
  pdfsubject={},
  pdfcreator={Emacs Org-mode version 7.9.3f}}

\begin{document}

\maketitle

\setcounter{tocdepth}{3}
\tableofcontents
\vspace*{1cm}
\section{Chapter 1}
\label{sec-1}
\subsection{Propositional Logic}
\label{sec-1-1}
\subsubsection{Converse Contrapositive and Inverse (p -> q)}
\label{sec-1-1-1}
\begin{itemize}

\item Coverse
\label{sec-1-1-1-1}%
\begin{itemize}

\item q $\rightarrow$  p
\label{sec-1-1-1-1-1}%
\end{itemize} % ends low level

\item Contrapositive
\label{sec-1-1-1-2}%
\begin{itemize}

\item q $\rightarrow$ \textlnot{}p
\label{sec-1-1-1-2-1}%
\end{itemize} % ends low level

\item Inverse
\label{sec-1-1-1-3}%
\begin{itemize}

\item \textlnot{}p $\rightarrow$ \textlnot{}q
\label{sec-1-1-1-3-1}%
\end{itemize} % ends low level
\end{itemize} % ends low level
\subsection{Applications of Propositional Logic}
\label{sec-1-2}
\subsubsection{Examples of turning sentences into propositional Logic.}
\label{sec-1-2-1}
\subsection{Propositional Equivalences}
\label{sec-1-3}
\subsubsection{Logical Equivalences}
\label{sec-1-3-1}


\begin{center}
\begin{tabular}{lll}
\hline
 p\wedge T                  &  p                                &  Identity Laws         \\
 p\vee F                    &  p                                &                        \\
\hline
 p\vee T                    &  T                                &  Domination Laws       \\
 p\wedge F                  &  F                                &                        \\
\hline
 p\vee p                    &  p                                &  Idempotent Laws       \\
 p\wedge p                  &  p                                &                        \\
\hline
 \textlnot{}(\textlnot{}p)  &  p                                &  Double Negation Laws  \\
\hline
 p\vee q                    &  q\vee p                          &  Commutative Laws      \\
 p\wedge q                  &  q \wedge p                       &                        \\
\hline
 (p\vee q)\vee r            &  p\vee (q\vee r)                  &  Associative Laws      \\
 (p\wedge q) \wedge r       &  p \wedge (q \wedge r)            &                        \\
\hline
 p\vee (q\wedge r)          &  (p\vee q) \wedge (p\vee r)       &  Distributive Laws     \\
 p \wedge (q\vee r)         &  (p\wedge q)\vee (p \wedge r)     &                        \\
\hline
 \textlnot{}(p\wedge q)     &  \textlnot{}p\vee \textlnot{}q    &  De Morgans Laws       \\
 \textlnot{}(p v q)         &  \textlnot{}p\wedge \textlnot{}q  &                        \\
\hline
 p\vee (p\wedge q)          &  p                                &  Absorption Laws       \\
 p \wedge (p\vee q)         &  p                                &                        \\
\hline
 p v \textlnot{}p           &  T                                &  Negation Laws         \\
 p\wedge \textlnot{}p       &  F                                &                        \\
\hline
\end{tabular}
\end{center}
\subsubsection{Logical Equivelances Involving Conditional Statements}
\label{sec-1-3-2}


\begin{center}
\begin{tabular}{ll}
\hline
 p \rightarrow q                            &  \textlnot{}p\vee q                       \\
\hline
 p \rightarrow q                            &  \textlnot{}q \rightarrow \textlnot{}p    \\
\hline
 p\vee q                                    &  \textlnot{}p \rightarrow q               \\
\hline
 p\wedge  q                                 &  \textlnot{}(p \rightarrow \textlnot{}q)  \\
\hline
 \textlnot{}(p \rightarrow q)               &  p\wedge q                                \\
\hline
 (p \rightarrow q)\wedge (p \rightarrow r)  &  p \rightarrow (q \wedge r)               \\
\hline
 (p \rightarrow r)\wedge (q \rightarrow r)  &  (p\vee q) \rightarrow r                  \\
\hline
 (p \rightarrow q)\vee (p \rightarrow r)    &  p \rightarrow (q\vee r)                  \\
\hline
 (p \rightarrow r)\vee (q \rightarrow r)    &  (p\wedge q) \rightarrow r                \\
\hline
\end{tabular}
\end{center}
\subsubsection{Logical Equivalences Involving Biconditional Statements}
\label{sec-1-3-3}


\begin{center}
\begin{tabular}{ll}
\hline
 p \leftrightarrow q               &  (p \rightarrow q)\wedge (q \rightarrow p)           \\
\hline
 p \leftrightarrow q               &  \textlnot{}p \leftrightarrow \textlnot{}q           \\
\hline
 p \leftrightarrow q               &  (p\wedge q)\vee (\textlnot{}p \wedge \textlnot{}q)  \\
\hline
 \textlnot{}(p \leftrightarrow q)  &  p \leftrightarrow \textlnot{}q                      \\
\hline
\end{tabular}
\end{center}
\subsection{Predicates and Quantifiers}
\label{sec-1-4}
\subsubsection{Quantifiers}
\label{sec-1-4-1}
\begin{itemize}

\item Universal Quantifer
\label{sec-1-4-1-1}%
\begin{itemize}

\item $\forall$(x) P(x)
\label{sec-1-4-1-1-1}%
\begin{itemize}

\item Definition: For all x in the universe P(x) is true;
\label{sec-1-4-1-1-1-1}%

\item Negation
\label{sec-1-4-1-1-1-2}%
\begin{itemize}

\item \textlnot{}$\forall$(x) P(x) can also be written as $\exists$\textlnot{} P(x)
\label{sec-1-4-1-1-1-2-1}%

\item There exists an x such that P(x) is false
\label{sec-1-4-1-1-1-2-2}%
\end{itemize} % ends low level
\end{itemize} % ends low level
\end{itemize} % ends low level

\item Existential Quantifer
\label{sec-1-4-1-2}%
\begin{itemize}

\item $\exists$(x) P(x)
\label{sec-1-4-1-2-1}%
\begin{itemize}

\item Definition: For all x in the universe P(x) is true for at least one x;
\label{sec-1-4-1-2-1-1}%

\item Negation
\label{sec-1-4-1-2-1-2}%
\begin{itemize}

\item \textlnot{}$\exists$(x) P(x) can also be written as $\forall$\textlnot{} P(x)
\label{sec-1-4-1-2-1-2-1}%

\item All x in the universe make P(x) false
\label{sec-1-4-1-2-1-2-2}%
\end{itemize} % ends low level
\end{itemize} % ends low level
\end{itemize} % ends low level
\end{itemize} % ends low level
\subsection{Nested Quantifers}
\label{sec-1-5}
\subsubsection{Nested Quantifiers can be used when there are multiple variables such as x and y}
\label{sec-1-5-1}
\begin{itemize}

\item $\forall$ (x) $\exists$ (y) (x + y = 0)
\label{sec-1-5-1-1}%
\end{itemize} % ends low level
\subsubsection{Quantification of Two Variables}
\label{sec-1-5-2}


\begin{center}
\begin{tabular}{lll}
\hline
 Statement                       &  When True                       &  When False                        \\
\hline
 \forall (x) \forall (y) P(x,y)  &  P(x,y) is true for every pair   &  When there is a x,y               \\
                                 &                                  &  for which P(x,y) is false         \\
\hline
 \forall (x) \exists (y) P(x,y)  &  For every x there is a y        &  When there is an x such           \\
                                 &  for which P(x,y) is true        &  that P(x,y) is false for every y  \\
\hline
 \exists (x) \forall (y) P(x,y)  &  There is an x for which P(x,y)  &  When for every x there is a y     \\
                                 &  is true for every y             &  for which P(x,y) is false         \\
\hline
 \exists (x) \exists (y) P(x,y)  &  There is a pair for x,y         &  P(x,y) is false for               \\
 \exists (y) \exists (x) P(x,y)  &  for which P(x,y) is true        &  every pair of x and y             \\
\hline
\end{tabular}
\end{center}
\subsection{Rules of Inference}
\label{sec-1-6}
\subsubsection{Rules of inference}
\label{sec-1-6-1}


\begin{center}
\begin{tabular}{lll}
\hline
 Rule of Inference  &  Tautology                                                       &  Name            \\
\hline
 p                  &  (p\wedge (p \rightarrow q)) \rightarrow q                       &  Modus Ponens    \\
 p \rightarrow q    &                                                                  &                  \\
 ---------------    &                                                                  &                  \\
 q                  &                                                                  &                  \\
\hline
 \textlnot q        &  (\textlnot q \wedge (p \rightarrow q)) \rightarrow \textlnot p  &  Modus Tollens   \\
 p \rightarrow q    &                                                                  &                  \\
 ---------------    &                                                                  &                  \\
 \textlnot p        &                                                                  &                  \\
\hline
 p \rightarrow q    &  ((p \vee q) \wedge \textlnot p) \rightarrow q                   &  Hypothetical    \\
 q \rightarrow r    &                                                                  &  Syllogism       \\
 ---------------    &                                                                  &                  \\
 p  \rightarrow r   &                                                                  &                  \\
\hline
 p\vee q            &  p \rightarrow (p \vee q)                                        &  Disjunctive     \\
 \textlnot p        &                                                                  &  Syllogism       \\
 ---------------    &                                                                  &                  \\
 q                  &                                                                  &                  \\
\hline
 p                  &  p \rightarrow (p\vee q)                                         &  Addition        \\
 ---------------    &                                                                  &                  \\
 p\vee q            &                                                                  &                  \\
\hline
 p\wedge q          &  (p\wedge q) \rightarrow p                                       &  Simplification  \\
 ---------------    &                                                                  &                  \\
 p                  &                                                                  &                  \\
\hline
 p                  &  ((p)\wedge(q)) \rightarrow (p \wedge q)                         &  Conjunction     \\
 q                  &                                                                  &                  \\
 ---------------    &                                                                  &                  \\
 p\wedge q          &                                                                  &                  \\
\hline
 p\vee q            &  ((p\vee q)\wedge (\textlnot p\vee r))\rightarrow (q\vee r)      &  Conjucntion     \\
 \textlnot p\vee r  &                                                                  &                  \\
 ---------------    &                                                                  &                  \\
 q\vee r            &                                                                  &                  \\
\hline
\end{tabular}
\end{center}
\subsubsection{Rules of Inference for Quantifies Statements}
\label{sec-1-6-2}


\begin{center}
\begin{tabular}{ll}
\hline
 Rules of Inference       &  Name                        \\
\hline
 \forall (x) P(x)         &  Universal Instantiation     \\
 ----------------         &                              \\
 P(c)                     &                              \\
\hline
 P(x) for an Arbitrary c  &  Universal Generalization    \\
 ----------------         &                              \\
 \forall(x) P(x)          &                              \\
\hline
 \exists (x) P(x)         &  Existential Instantiation   \\
 ----------------         &                              \\
 P(c) for some element c  &                              \\
\hline
 P(x) for some element c  &  Existential Generalization  \\
 ----------------         &                              \\
 \exists P(x)             &                              \\
\hline
\end{tabular}
\end{center}
\subsection{Introduction to Proofs}
\label{sec-1-7}
\subsubsection{There are 3 main methods for proofs.}
\label{sec-1-7-1}
\begin{itemize}

\item Direct Proof
\label{sec-1-7-1-1}%
\begin{itemize}

\item Construct a conditional statement p $\rightarrow$ q
\label{sec-1-7-1-1-1}%

\item Assume p to be true
\label{sec-1-7-1-1-2}%

\item Use rules of inference to then show that when p is true q must be true (p true and q false can not happen)
\label{sec-1-7-1-1-3}%
\end{itemize} % ends low level

\item Proof by Contraposition
\label{sec-1-7-1-2}%
\begin{itemize}

\item Construct a conditional statement p $\rightarrow$ q
\label{sec-1-7-1-2-1}%

\item Set up the contrapositive to be \textlnot q $\rightarrow$ \textlnot p
\label{sec-1-7-1-2-2}%

\item Prove that if \textlnot q is true then \textlnot p has to be true
\label{sec-1-7-1-2-3}%
\end{itemize} % ends low level

\item Proof by Contradiction
\label{sec-1-7-1-3}%
\begin{itemize}

\item If we want to prove p is true set up the contradiction to be \textlnot p $\rightarrow$ q
\label{sec-1-7-1-3-1}%

\item Because q is false but \textlnot p $\rightarrow$ q is true we know \textlnotp is false which means p is true
\label{sec-1-7-1-3-2}%

\item You are assuming your premise to be false then attempting to show that the conditional statement is then false
\label{sec-1-7-1-3-3}%
\end{itemize} % ends low level
\end{itemize} % ends low level
\subsection{Proof methods and Strategy}
\label{sec-1-8}
\subsubsection{Exhaustive Proof}
\label{sec-1-8-1}
\begin{itemize}

\item Sometimes w cannot prove a theorem using a single argument that holds for all possible cases.
\label{sec-1-8-1-1}%

\item Proofs by exhaustion use proof by cases for every element and check examples.
\label{sec-1-8-1-2}%

\item Example
\label{sec-1-8-1-3}%
\begin{itemize}

\item Prove that (n+1)$^3$ $\ge$ 3$^n$ if n is a positive integer $\le$ 4
\label{sec-1-8-1-3-1}%

\item To do this test for n = \{1,2,3,4\}
\label{sec-1-8-1-3-2}%

\item Since for n = \{1,2,3,4\} 3$^n$ is greater we proved this statement through Proof by Exhaustion
\label{sec-1-8-1-3-3}%
\end{itemize} % ends low level

\item Proof by Cases
\label{sec-1-8-1-4}%
\begin{itemize}

\item A proof by cases must cover all possible cases that arise in a theorem.
\label{sec-1-8-1-4-1}%

\item Example
\label{sec-1-8-1-4-2}%
\begin{itemize}

\item Prove that if n is an integer than n$^2$ $\ge$ n
\label{sec-1-8-1-4-2-1}%

\item We check 3 cases
\label{sec-1-8-1-4-2-2}%

\item 1) n = 0; 0$^2$ = 0 which follows n$^2$ $\ge$ n
\label{sec-1-8-1-4-2-3}%

\item 2) n $\ge$ 1; Multiply both sides of the inequality n $\ge$ 1 by the positive integer n. We get n$^2$ > n*1 for n $\ge$ 1
\label{sec-1-8-1-4-2-4}%

\item 3) n $\le$ 1; Since n$^2$ $\ge$ 0 it follows n$^2$ $\ge$ n
\label{sec-1-8-1-4-2-5}%
\end{itemize} % ends low level
\end{itemize} % ends low level

\item Leveraging Proof by Cases
\label{sec-1-8-1-5}%
\begin{itemize}

\item 
%
\end{itemize} % ends low level

\item Existence Proofs
\label{sec-1-8-1-6}%
\begin{itemize}

\item 
%
\end{itemize} % ends low level

\item Proof Stratagies
\label{sec-1-8-1-7}%
\begin{itemize}

\item Forward and Backward Reasoning
\label{sec-1-8-1-7-1}%

\item Adapting Existing Proofs
\label{sec-1-8-1-7-2}%

\item Looking for Counterexamples
\label{sec-1-8-1-7-3}%
\end{itemize} % ends low level
\end{itemize} % ends low level
\section{Chapter 2}
\label{sec-2}
\subsection{Sets}
\label{sec-2-1}
\subsubsection{Definition}
\label{sec-2-1-1}
\begin{itemize}

\item A set is an unordered collection of objects. a $\in$ A
\label{sec-2-1-1-1}%

\item Set builder notation
\label{sec-2-1-1-2}%
\begin{itemize}

\item Set builder notation is used to describe a set.
\label{sec-2-1-1-2-1}%

\item O = \{x | x is a positive integer less than 10\}
\label{sec-2-1-1-2-2}%

\item O = \{x | x < 10 $\wedge$ 2x+1 $\in$ Z+ \}
\label{sec-2-1-1-2-3}%
\end{itemize} % ends low level
\end{itemize} % ends low level
\subsubsection{Number sets}
\label{sec-2-1-2}
\begin{itemize}

\item N = \{0,1,2,..,\} - Set of Natural Numbers
\label{sec-2-1-2-1}%

\item Z = \{..., -1,0,1,\ldots{},\} - Set of Intgers
\label{sec-2-1-2-2}%

\item Z+ = \{1,2,3,\ldots{},\} - Set of Positive Integers
\label{sec-2-1-2-3}%

\item Q = \{p/q | p $\in$ Z, q /in Z, and q $\ne$ 0\} - Set of Rational Numbers
\label{sec-2-1-2-4}%

\item R = Set of Real Numbers
\label{sec-2-1-2-5}%

\item R+ = Set of positive Real Numbers
\label{sec-2-1-2-6}%

\item C = Set of complex numbers
\label{sec-2-1-2-7}%
\end{itemize} % ends low level
\subsubsection{Interval Notation}
\label{sec-2-1-3}


\begin{center}
\begin{tabular}{ll}
\hline
 [a,b]  &  \{a \le x \le b\}              \\
\hline
 [a,b)  &  \{a \le x\} \lessthan b \}     \\
\hline
 (a,b]  &  \{a \lessthan x \le b \}       \\
\hline
 (a,b)  &  \{a \lessthan x \lessthan b\}  \\
\hline
\end{tabular}
\end{center}
\subsubsection{Empty Set}
\label{sec-2-1-4}
\begin{itemize}

\item Denoted by $\emptyset$
\label{sec-2-1-4-1}%

\item All sets contain the empty set although it does not count as a element when measuring cardinality
\label{sec-2-1-4-2}%
\end{itemize} % ends low level
\subsubsection{Subsets}
\label{sec-2-1-5}
\begin{itemize}

\item The set A is a subset of B iff every element of A is also in B. Denoted by A \subseteq B
\label{sec-2-1-5-1}%
\end{itemize} % ends low level
\subsubsection{Proper Subset}
\label{sec-2-1-6}
\begin{itemize}

\item The set A is a subset of B iff every element of A is also in B but A $\neq$ B. Denoted by A $\subset$ B
\label{sec-2-1-6-1}%
\end{itemize} % ends low level
\subsubsection{Set Cardinality}
\label{sec-2-1-7}
\begin{itemize}

\item If S is a set the cardinality of the set denoted by |S| is the number of UNIQUE elements.
\label{sec-2-1-7-1}%
\begin{itemize}

\item Example S = \{1,2,3,3,4,4,5\} |S| = 5
\label{sec-2-1-7-1-1}%
\end{itemize} % ends low level
\end{itemize} % ends low level
\subsubsection{Power Sets}
\label{sec-2-1-8}
\begin{itemize}

\item Definition: Given the set S the power set of S i the set of all subsets of the set S. The powerset of S is denoted by P(S)
\label{sec-2-1-8-1}%
\begin{itemize}

\item Example: What is the powerset of \{0,1,2\}
\label{sec-2-1-8-1-1}%

\item P(S) = \{$\emptyset$, \{0\}, \{1\}, \{2\}, \{0,1\}, \{0,2\}, \{1,2\},\{0,1,2\},\}
\label{sec-2-1-8-1-2}%
\end{itemize} % ends low level
\end{itemize} % ends low level
\subsubsection{Cartesian Product}
\label{sec-2-1-9}
\begin{itemize}

\item Definition: The ordered n-tuple (a1,a2..an) is the ordered collection that has a1 as its first element, a2 as its second..and an as its nth element.
\label{sec-2-1-9-1}%

\item A x B = \{(a,b) | a $\in$ A $\wedge$ b $\in$ B\}
\label{sec-2-1-9-2}%

\item Example: What is the cartesian Product of A = \{1,2\} B = \{a,b,c\}
\label{sec-2-1-9-3}%
\begin{itemize}

\item A x B = \{(1, a), (1, b), (1, c), (2, a), (2, b), (2, c)\}
\label{sec-2-1-9-3-1}%

\item B x A = \{(a, 1), (a, 2), (b, 1), (b, 2), (c, 1), (c, 2)\}
\label{sec-2-1-9-3-2}%
\end{itemize} % ends low level

\item Example: What is the cartesian Product of A = \{0,1\} B = \{1,2\} C = \{0,1,2\}
\label{sec-2-1-9-4}%
\begin{itemize}

\item A × B × C = \{(0, 1, 0), (0, 1, 1), (0, 1, 2), (0, 2, 0), (0, 2, 1), (0, 2, 2),(1, 1, 0), (1, 1, 1), (1, 1, 2), (1, 2, 0), (1, 2, 1), (1, 2, 2)\}.
\label{sec-2-1-9-4-1}%
\end{itemize} % ends low level
\end{itemize} % ends low level
\subsubsection{Truth Set}
\label{sec-2-1-10}
\begin{itemize}

\item The truth set for a predicate P, and domain D the truth set of P is \{x $\in$ D | P(x)\}
\label{sec-2-1-10-1}%

\item Another way to phrase this is the Truth set is the set that makes a predicate true in the domain.
\label{sec-2-1-10-2}%
\end{itemize} % ends low level
\subsection{Set Operations}
\label{sec-2-2}
\subsubsection{Union}
\label{sec-2-2-1}
\begin{itemize}

\item If A and B are sets the union of the sets A and B, denoted by A $\cup$ B is the set that contains elements in either or both A and B
\label{sec-2-2-1-1}%

\item A $\cup$ B = \{x | x $\in$ A $\vee$ x $\in$ B\}
\label{sec-2-2-1-2}%
\end{itemize} % ends low level
\subsubsection{Intersection}
\label{sec-2-2-2}
\begin{itemize}

\item If A and B are sets, the intersection of the sets A and B, denoted by A $\cap$ B, is the set containing elements in both A and B
\label{sec-2-2-2-1}%

\item A $\cap$ B = \{x | x $\in$ A $\wedge$ x $\in$ B\}.
\label{sec-2-2-2-2}%
\end{itemize} % ends low level
\subsubsection{Disjoint}
\label{sec-2-2-3}
\begin{itemize}

\item Two sets are considered disjoint if they have no intersection or their intersection set is empty
\label{sec-2-2-3-1}%
\end{itemize} % ends low level
\subsubsection{Difference}
\label{sec-2-2-4}
\begin{itemize}

\item If A and B are sets, the difference of A and B denoted by A - B is the set elements in A that are not in B
\label{sec-2-2-4-1}%

\item A - B = \{x | x $\in$ A $\wedge$ x $\notin$ B \}
\label{sec-2-2-4-2}%
\end{itemize} % ends low level
\subsubsection{Complement}
\label{sec-2-2-5}
\begin{itemize}

\item If U is the universal set, the complement of the set A denoted by A(bar) is the set of U - A or all the elements in the universe not in A
\label{sec-2-2-5-1}%

\item A(bar) = \{x $\in$ U | x $\notin$ A\}
\label{sec-2-2-5-2}%
\end{itemize} % ends low level
\subsubsection{Set Identities}
\label{sec-2-2-6}


\begin{center}
\begin{tabular}{lll}
\hline
 p\wedge U               &  A                                &  Identity Laws         \\
 p\vee \emptyset         &  A                                &                        \\
\hline
 A\vee T                 &  U                                &  Domination Laws       \\
 A\wedge F               &  \emptyset                        &                        \\
\hline
 A\vee A                 &  A                                &  Idempotent Laws       \\
 A\wedge A               &  A                                &                        \\
\hline
 (A(bar)bar)             &  A                                &  Double Negation Laws  \\
\hline
 A\vee B                 &  B\vee A                          &  Commutative Laws      \\
 A\wedge B               &  B \wedge A                       &                        \\
\hline
 (A\vee B)\vee C         &  p\vee (B\vee C)                  &  Associative Laws      \\
 (A\wedge B) \wedge C    &  A\wedge (B \wedge C)             &                        \\
\hline
 p\vee (B\wedge C)       &  (p\vee B) \wedge (p\vee C)       &  Distributive Laws     \\
 A\wedge (B\vee C)       &  (p\wedge B)\vee (A\wedge C)      &                        \\
\hline
 \textlnot{}(p\wedge B)  &  \textlnot{}p\vee \textlnot{}B    &  De Morgans Laws       \\
 \textlnot{}(Av B)       &  \textlnot{}p\wedge \textlnot{}B  &                        \\
\hline
 A\vee (p\wedge B)       &  A                                &  Absorption Laws       \\
 A\wedge (p\vee B)       &  A                                &                        \\
\hline
 A V A(bar)              &  U                                &  Complement Laws       \\
 A \& A(bar)             &  Empty Set                        &                        \\
\hline
\end{tabular}
\end{center}
\subsection{Functions}
\label{sec-2-3}
\subsubsection{Functions}
\label{sec-2-3-1}
\begin{itemize}

\item Let A and B be nonempty sets. A function f from A to B is an assignment of exactly one element of B to each element of A
\label{sec-2-3-1-1}%

\item This also means that every element of A gets mapped to at exactly one set of B
\label{sec-2-3-1-2}%

\item Everything in set A gets mapped to an element of set B
\label{sec-2-3-1-3}%

\item (f1+f2)(x) = f1(x) + f2(x)
\label{sec-2-3-1-4}%
\end{itemize} % ends low level
\subsubsection{Image}
\label{sec-2-3-2}
\begin{itemize}

\item If f is a function that maps A to B, elements of B are called the image and the elements of A are called the preimage.
\label{sec-2-3-2-1}%
\end{itemize} % ends low level
\subsubsection{One to One (Injective)}
\label{sec-2-3-3}
\begin{itemize}

\item If f is a function that maps A to B, each a $\in$ A maps to a unique b $\in$ B
\label{sec-2-3-3-1}%
\end{itemize} % ends low level
\subsubsection{Onto}
\label{sec-2-3-4}
\begin{itemize}

\item If f is a function that maps A to B, the function is onto if all elements of the codomain are mapped to from values of A
\label{sec-2-3-4-1}%
\end{itemize} % ends low level
\subsubsection{One to one correspondence}
\label{sec-2-3-5}
\begin{itemize}

\item If f is a function that maps A to b, it has one to one correspondance if it is both one to one and onto
\label{sec-2-3-5-1}%
\end{itemize} % ends low level
\subsubsection{Inverse Functions}
\label{sec-2-3-6}
\begin{itemize}

\item If a function f has a one to one correspondance it has an inverse function f-1 that maps the codomain to the domain.b
\label{sec-2-3-6-1}%
\end{itemize} % ends low level
\subsubsection{Composition}
\label{sec-2-3-7}
\begin{itemize}

\item (f o f-1)(x) = x
\label{sec-2-3-7-1}%

\item (f o g)(x) = f(g(x))
\label{sec-2-3-7-2}%

\item (f + g)(x) = f(x) + g(x)
\label{sec-2-3-7-3}%

\item (f * g)(x) = f(x) * g(x)
\label{sec-2-3-7-4}%
\end{itemize} % ends low level
\subsection{Sequences and Summations}
\label{sec-2-4}
\subsubsection{Definition: Sequence is a function from a subset of the set of integers to a set S. They're also an ordered set.}
\label{sec-2-4-1}
\subsubsection{Example:}
\label{sec-2-4-2}
\begin{itemize}

\item \{1,2,3,4,5,6,\ldots{},\}
\label{sec-2-4-2-1}%
\end{itemize} % ends low level
\subsubsection{Geometric Sequence}
\label{sec-2-4-3}
\begin{itemize}

\item a, ar, ar$^2$,\ldots{},ar$^n$
\label{sec-2-4-3-1}%

\item Where a is the inital term and r is a ration
\label{sec-2-4-3-2}%
\end{itemize} % ends low level
\subsubsection{Arithmetic Progression}
\label{sec-2-4-4}
\begin{itemize}

\item a, a+d, a+2d,\ldots{},a+nd
\label{sec-2-4-4-1}%

\item Where a is the initial term and d is the common difference
\label{sec-2-4-4-2}%
\end{itemize} % ends low level
\subsubsection{Recurrence Relations}
\label{sec-2-4-5}
\begin{itemize}

\item A reccurence relation for the sequence \{an\} is an equation that expresses an in terms of one or more of the previous terms.
\label{sec-2-4-5-1}%

\item Example:
\label{sec-2-4-5-2}%
\begin{itemize}

\item a0 = 0;
\label{sec-2-4-5-2-1}%

\item a1 = a(n-1)+1
\label{sec-2-4-5-2-2}%
\end{itemize} % ends low level
\end{itemize} % ends low level
\subsubsection{We say that we have solved a recurrence relation together with the initial condition when we find a formula. We call this the closed formula.}
\label{sec-2-4-6}
\begin{itemize}

\item Example:
\label{sec-2-4-6-1}%
\begin{itemize}

\item DO THIS
\label{sec-2-4-6-1-1}%
\end{itemize} % ends low level
\end{itemize} % ends low level
\subsubsection{Summations}
\label{sec-2-4-7}


\begin{center}
\begin{tabular}{ll}
\hline
 \sum (k = 0 \rightarrow n) ar^k                      &  ar^(n+1) -a/(r-1) r \neq 1  \\
\hline
 \sum (k = 1 \rightarrow n) k                         &  n(n+1)/2                    \\
\hline
 \sum (k = 1 \rightarrow n) k^2                       &  n(n+1)(2n+1)/6              \\
\hline
 \sum (k = 1 \rightarrow n) k^3                       &  n^2(n+1)^2/4                \\
\hline
 \sum (k = 0 \rightarrow \inf) x^k abs(x) \le 1       &  1/(1-x)                     \\
\hline
 \sum (k = 0 \rightarrow \inf) kx^(k-1) abs(x) \le 1  &  1/(1-x)^2                   \\
\hline
\end{tabular}
\end{center}
\subsection{Cardinality of Sets}
\label{sec-2-5}
\subsubsection{Definition: The sets A and B have the same cardinality iff there is a one to one correspondance from A to B, we write |A| = |B|}
\label{sec-2-5-1}
\subsubsection{Countability}
\label{sec-2-5-2}
\begin{itemize}

\item A set that is either finite or has the same cardinality as the set of positive integers is called countable..
\label{sec-2-5-2-1}%

\item If A and B are both countable sets then A $\cup$ B is also countable.
\label{sec-2-5-2-2}%



\end{itemize} % ends low level

\end{document}
