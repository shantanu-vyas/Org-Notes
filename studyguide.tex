% Created 2016-10-17 Mon 02:52
\documentclass[11pt]{article}
\usepackage[utf8]{inputenc}
\usepackage[T1]{fontenc}
\usepackage{fixltx2e}
\usepackage{graphicx}
\usepackage{longtable}
\usepackage{float}
\usepackage{wrapfig}
\usepackage{soul}
\usepackage{textcomp}
\usepackage{marvosym}
\usepackage{wasysym}
\usepackage{latexsym}
\usepackage{amssymb}
\usepackage{hyperref}
\tolerance=1000
\providecommand{\alert}[1]{\textbf{#1}}

\title{studyguide}
\author{Shantanu Vyas}
\date{\today}
\hypersetup{
  pdfkeywords={},
  pdfsubject={},
  pdfcreator={Emacs Org-mode version 7.9.3f}}

\begin{document}

\maketitle

\setcounter{tocdepth}{3}
\tableofcontents
\vspace*{1cm}
\section{Chapter 1}
\label{sec-1}
\subsection{Propositional Logic}
\label{sec-1-1}
\subsubsection{Converse Contrapositive and Inverse (p -> q)}
\label{sec-1-1-1}
\begin{itemize}

\item Coverse
\label{sec-1-1-1-1}%
\begin{itemize}

\item q $\rightarrow$  p
\label{sec-1-1-1-1-1}%
\end{itemize} % ends low level

\item Contrapositive
\label{sec-1-1-1-2}%
\begin{itemize}

\item -q $\rightarrow$ -p
\label{sec-1-1-1-2-1}%
\end{itemize} % ends low level

\item Inverse
\label{sec-1-1-1-3}%
\begin{itemize}

\item -p $\rightarrow$ -q
\label{sec-1-1-1-3-1}%
\end{itemize} % ends low level
\end{itemize} % ends low level
\subsection{Applications of Propositional Logic}
\label{sec-1-2}
\subsubsection{Examples of turning sentences into propositional Logic.}
\label{sec-1-2-1}
\subsection{Propositional Equivalences}
\label{sec-1-3}
\subsubsection{Logical Equivalences}
\label{sec-1-3-1}


\begin{center}
\begin{tabular}{lll}
\hline
 p\wedge T             &  p                             &  Identity Laws         \\
 p\vee F               &  p                             &                        \\
\hline
 p\vee T               &  T                             &  Domination Laws       \\
 p\wedge F             &  F                             &                        \\
\hline
 p\vee p               &  p                             &  Idempotent Laws       \\
 p\wedge p             &  p                             &                        \\
\hline
 -(-p)                 &  p                             &  Double Negation Laws  \\
\hline
 p\vee q               &  q\vee p                       &  Commutative Laws      \\
 p\wedge q             &  q \wedge p                    &                        \\
\hline
 (p\vee q)\vee r       &  p\vee (q\vee r)               &  Associative Laws      \\
 (p\wedge q) \wedge r  &  p \wedge (q \wedge r)         &                        \\
\hline
 p\vee (q\wedge r)     &  (p\vee q) \wedge (p\vee r)    &  Distributive Laws     \\
 p \wedge (q\vee r)    &  (p\wedge q)\vee (p \wedge r)  &                        \\
\hline
 -(p\wedge q)          &  -p\vee -q                     &  De Morgans Laws       \\
 -(p v q)              &  -p\wedge -q                   &                        \\
\hline
 p\vee (p\wedge q)     &  p                             &  Absorption Laws       \\
 p \wedge (p\vee q)    &  p                             &                        \\
\hline
 p v -p                &  T                             &  Negation Laws         \\
 p\wedge -p            &  F                             &                        \\
\hline
\end{tabular}
\end{center}
\subsubsection{Logical Equivelances Involving Conditional Statements}
\label{sec-1-3-2}


\begin{center}
\begin{tabular}{ll}
\hline
 p \rightarrow q                            &  -p\vee q                    \\
\hline
 p \rightarrow q                            &  -q \rightarrow -p           \\
\hline
 p\vee q                                    &  -p \rightarrow q            \\
\hline
 p\wedge  q                                 &  -(p \rightarrow -q)         \\
\hline
 -(p \rightarrow q)                         &  p\wedge q                   \\
\hline
 (p \rightarrow q)\wedge (p \rightarrow r)  &  p \rightarrow (q \wedge r)  \\
\hline
 (p \rightarrow r)\wedge (q \rightarrow r)  &  (p\vee q) \rightarrow r     \\
\hline
 (p \rightarrow q)\vee (p \rightarrow r)    &  p \rightarrow (q\vee r)     \\
\hline
 (p \rightarrow r)\vee (q \rightarrow r)    &  (p\wedge q) \rightarrow r   \\
\hline
\end{tabular}
\end{center}
\subsubsection{Logical Equivalences Involving Biconditional Statements}
\label{sec-1-3-3}


\begin{center}
\begin{tabular}{ll}
\hline
 p \leftrightarrow q     &  (p \rightarrow q)\wedge (q \rightarrow p)  \\
\hline
 p \leftrightarrow q     &  -p \leftrightarrow -q                      \\
\hline
 p \leftrightarrow q     &  (p\wedge q)\vee (-p \wedge -q)             \\
\hline
 -(p \leftrightarrow q)  &  p \leftrightarrow -q                       \\
\hline
\end{tabular}
\end{center}
\subsection{Predicates and Quantifiers}
\label{sec-1-4}
\subsubsection{Quantifiers}
\label{sec-1-4-1}
\begin{itemize}

\item Universal Quantifer
\label{sec-1-4-1-1}%
\begin{itemize}

\item ∀(x) DX
\label{sec-1-4-1-1-1}%

\item \guillemotleft{}
\label{sec-1-4-1-1-2}%
\end{itemize} % ends low level
\end{itemize} % ends low level
\subsection{Nested Quantifers}
\label{sec-1-5}
\subsection{Rules of Inference}
\label{sec-1-6}
\subsection{Introduction to Proofs}
\label{sec-1-7}
\subsection{Proof methods and Strategy}
\label{sec-1-8}
\section{Chapter 2}
\label{sec-2}
\subsection{Sets}
\label{sec-2-1}
\subsection{Set Operations}
\label{sec-2-2}
\subsection{Functions}
\label{sec-2-3}
\subsection{Sequences and Summations}
\label{sec-2-4}
\subsection{Cardinality of Sets}
\label{sec-2-5}
\subsection{Matrices}
\label{sec-2-6}

\end{document}
